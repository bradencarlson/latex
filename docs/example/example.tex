\documentclass[nopresentation]{notes}
\usepackage{amsmath,amstext,amssymb}
\usepackage{hyperref}

\begin{document}

\section{Linear Equations and Matrices}

\subsection{Definitions and Examples}

\content{
\begin{definition} A \textbf{linear equation} in variables $x_1,x_2,\dots x_n$ is an equation that 
can be written in the form 
$$ a_1x_1+a_2x_2+\cdots +a_nx_n=b $$
where $a_1,a_2,\dots,a_n, $ and $b$ are real or complex numbers. The numbers $a_1,a_2,\dots, a_n$ are 
called the \textbf{coefficients} of the equation.
\end{definition}
}{ % presentation
}{ % nopresentation
}{% comments
}

\content{
\begin{example} Determine whether each of the following is a linear equation. 
\begin{enumerate}
		\item $ 5x-3y=0 $
		\item $ (1+i)x_1^2-2x_1x_2+3x_2=i $, where $i^2=-1$.
		\item $ \frac{1}{x}-y=5 $
\end{enumerate}
\end{example}
}{ % presentation 
\vspace{2in}
}{% nopresentation
Solution: 
\begin{enumerate}
        \item This is a linear equation since it is of the form in the definition above.
        \item This is \textbf{not} a linear equation.  The presence of the $x_1^2$, as well 
                as the term $x_1x_2$ show that it is not of the form in the definition, and in fact, 
                cannot be put into the form above, so this is not linear.
        \item This is also not a linear equation.  Similarly to number 2, this equation cannot 
                be manipulated into the correct form for a linear equation, so it is not linear. 
\end{enumerate}
}{% comments
Proving that the equations cannot be put into linear form is not needed here for
this class. 
}

\content{
\begin{definition} A system of linear equations is called a \textbf{linear system}.
\end{definition}

\begin{example} The following is an example of a linear system.
\begin{eqnarray*}
		5x-3y+z &=&0 \\
		2x+y-2z &=& 1
\end{eqnarray*}
\end{example}
}{% presentation
}{% nopresentation
}{% comments
}

\content{
\begin{definition} Given a linear system in variables $x_1,x_2,\dots, x_n$, we say that 
$(s_1,s_2,\dots, s_n)$ is a \textbf{solution} to the linear system if upon substituting $s_1$ for $x_1$, $s_2$ for 
$x_2$, \dots, and $s_n$ for $x_n$, each linear equation in the system is satisfied. The set of all 
solutions to a linear system is called a \textbf{solution set}.
\end{definition}

\begin{definition} Given a linear system, we say that the system is \textbf{consistent} if there exists at least 
one solution.  If a linear system has no solution, then we say that the system is \textbf{inconsistent}. 
\end{definition}
}{% presentation
}{% nopresentation
}{% comments
}

\content{
\begin{example} In the above linear system,
\begin{eqnarray*}
		5x-3y+z &=&0 \\
		2x+y-2z &=& 1
\end{eqnarray*}
It is easily verified that $(\frac{3}{11},\frac{5}{11},0)$ is a solution 
by substituting $x=\frac{3}{11}$, $y=\frac{5}{11}$, and $z=0$.
\end{example}
}{% presentation
\vspace{2in}
}{% nopresentation
Note that 
$$ 5\left(\frac{3}{11}\right) - 3\left(\frac{5}{11}\right) +0 = 0 $$
and 
$$ 2\left(\frac{3}{11}\right) +\frac{5}{11}-0 = \frac{11}{11} = 1. $$
So this system is consistent, since it has at least one solution.
}{% comments
}

\subsection{Matrices}

\content{
Consider the following linear system:

\begin{equation}\label{eqn:linear-sys}
		\begin{array}{rrrr}
		 5x & -3y & +2z &= 0 \\
		-x & +y & &= 2 \\
		 &-y & +z &= 1
\end{array}
\end{equation}

We can summarize the left hand side of this linear system by writing all the coefficients 
of each variable in a rectangular array like so:

\begin{equation*}\label{eqn:coeff-matrix}
		\left(\begin{array}{rrr}
				5 & -3 & 2   \\
				-1 & 1 & 0  \\
				0 & -1 & 1 
		\end{array}\right)
\end{equation*}
}{% presentation
}{% nopresentation
Note that the first column of this array corresponds to the coefficients 
of the variable $x$, the second column 
is the coefficients of the variable $y$, and the last column is the coefficients 
of the variable $z$.  Similarly, 
each row corresponds to one of the equations of the system. 
}{% comments
Mention that the first column corresponds to the coefficients of the first
variable, the second column with those of the second variable, and so on. 
}

\content{
\begin{definition}
This is called the \textbf{coefficient matrix} of the system (\ref{eqn:linear-sys}). 
We could also summarize this linear system by adding the three numbers 
on the opposite side of the equals sign as another column, as so

\begin{equation*}
		\left(\begin{array}{rrrr}
				5 & -3 & 2 & 0  \\
				-1 & 1 & 0 & 2 \\
				0 & -1 & 1 & 1
		\end{array}\right)
\end{equation*}

This is called the \textbf{augmented matrix} of the linear system (\ref{eqn:linear-sys}).

When talking about a matrix, if a matrix $M$ has $m$ rows and $n$ columns,
then we say that $M$ is of \textbf{size}
$m\times n$ (read ``$m$ by $n$'').
\end{definition}

}{% presentation 
\vspace{2in}
}{% nopresentation
}{% comments
}

\content{
\begin{example} Find a solution to the following linear system using the elimination method, after each 
step, write the associated matrix of the linear system.
\begin{equation*}
		\begin{array}{rrrr}
				x & -2y & +z & = 0 \\
				  & 2y & -8z & = 8 \\
				5x & & -5z & = 10
		\end{array}
\end{equation*}
\end{example}

}{% presentation 
\vspace{2in}
}{% nopresentation
		After some algebra, we obtain the following system of linear equations, along with its corresponding augmented 
matrix:\\
\begin{center}
\begin{minipage}{0.3\textwidth}
\begin{align*}
		x-2y+z &= 0 & \\
		x-7z &= 8 & \\
				z&=-1
\end{align*}
\end{minipage}
\begin{minipage}{0.3\textwidth}
		\begin{equation*}
		\begin{pmatrix} 1 & -2 & 1 & 0 \\ 1 & 0 & -7 & 8 \\ 0 & 0 & 1 & -1 \end{pmatrix}
		\end{equation*}
\end{minipage}
\end{center} 
In the process of elimination, and solving the system, at each step when we write out the augmented matrix of the linear system, 
we see that the same operations we perform on the system are reflected in the matrix. This motivates the following definition.
}{% comments
The resulting matrix should be 
$$\begin{pmatrix} 1 & -2 & 1 & 0 \\ 1 & 0 & -7 & 8 \\ 0 & 0 & 1 & -1
\end{pmatrix}$$
For now, we stop because we know the value of $z$, and can then find $x$
and $y$. 
}

\content{
\begin{definition} (Elementary Row Operations) 
\begin{enumerate}
		\item (Multiply Row Add) One row is replaced by the sum of itself and a nonzero multiple of 
				another row.
		\item (Scale) One row is replaced by a multiple of itself.
		\item (Row Swap) Two rows swap positions.
\end{enumerate}
\end{definition}

\begin{definition} Two matrices are said to be \textbf{row equivalent} if there is some sequence of 
elementary row operations transforming one into the other. 
\end{definition}

\begin{theorem}
		If the augmented matrices of two linear systems are row equivalent, then the two 
systems have the same solution set.
\end{theorem}
}{% presentation
}{% nopresentation
}{% comments
}

\end{document}
